
\section{Introdução}
%Contexto geral de transformação
A maturidade alcançada por tecnologias como \textit{big data}, inteligência artificial, internet das coisas, computação em nuvem, dentre muitas outras, e a redução dos custos básicos de TI, resultante da disponibilização de infraestrutura como serviço e a disputa de mercado entre os grandes provedores de serviços de computação \cite{infraPriceDrop}, vem viabilizando e impulsionando um movimento de transformação digital em diversos ramos e organizações.

As organizações, sejam elas públicas ou privadas, de pequeno ou grande porte, em sua maioria já operam com o auxílio de um conjunto de soluções de \textit{sofware} que armazenam os dados de negócio, gerados tanto pela própria organização quanto por clientes e parceiros. Somam-se a esses dados, as informações publicamente disponíveis na internet, de modo que esse conjunto de dados esconde informações valiosíssimas que se bem utilizadas, podem significar a sucesso de um negócio \cite{digitalization}. 

Aliando-se, então, a disponibilidade dos dados, a redução no custo dos insumos de TI e a maturidade atingida por novas tecnologias, temos a construção de um cenário que vem impulsionando as iniciativas de desenvolvimento e implantação de soluções de dados por meio da aplicação de técnicas como \textit{Business Intelligence} (BI), que consiste na aquisição, tratamento de dados de negócio e armazenamento de dados de negócio, bem como o processo analítico sobre esses dados com o intuito de descoberta de padrões e tendências \cite{BIDef}.
%TODO: buscar referência de definição de BI e possível ajuste 

%Iniciativa da Residência no TRE
Neste contexto, uma das iniciativas do Tribunal Regional Eleitoral do Rio Grande do Norte (TRE-RN) para fomentar o desenvolvimento de solução de dados foi a implantação de uma turma de residência em tecnologia da informação em parceria com a Universidade Federal do Rio Grande do Norte (UFRN), onde uma das ênfases a serem trabalhadas no Tribunal era exatamente a de \textit{Business Intelligence}. 

Esse formato de residência em TI trata-se de um iniciativa pioneira da UFRN para, em parceria com orgão externos, formentar a formação de profissionais qualificados em tecnologia e técnicas inovadoras no âmbito da Tecnologia da Informação (TI). Neste programa, são selecionados alunos que cursarão um programa de pós graduação na Universidade e trabalharão alocados em projetos transformacionais dentro da instituição parceira, para ideação e desenvolvimento de soluções inovadoras por meio da aplicação de novas tecnologias.

%Objetivo do BI
Tratando-se especificamente da vertente de BI da turma de residência do TRE, e considerando que o TRE-RN não possuia modelo prévio, nem ferramentas voltadas para o desenvolvimento de produtos deste tipo, o objetivo principal da iniciativa foi a elaboração e implantação de um arquitetura de BI para desenvolvimento dos produtos composto por componentes de \textit{software} livre. 

%Arquitetura
Durante o curso desta residência, diversos \textit{softwares}, técnicas e tecnologias foram testadas e avaliadas pelo corpo técnico do Tribunal até que chegassemos em uma arquitetura, composto apenas de componentes gratuitos, que atendesse as necessidades e expectativas da equipe, resultando em uma solução capaz de suportar todo o processo de desenvolvimento de uma aplicação de BI. Esta arquitetura tem seus componentes hospedados por meio de containeres \textit{Docker} \cite{dockerDoc}, que já é um padrão de hospedagem de serviços do Orgão.

A arquitetura de BI passou a ser utilizado, após sua concepção, por uma quantidade restrita de usuários envolvidos no processo de desenvolvimento e homologação dos produtos e apresentou níveis de performance e disponibilidade dentro dos padrões esperandos, no entanto, ao ser submetido a uma carga de 50 a 60 usuários concorrêntes, passou-se a observar um alto nível de degradação de performance bem como eventos de indisponibilidade, principalmente no serviço de visualização desta arquitetura. 

Portanto, com o intuito de garantir mais robustez e escabilidade, este trabalho descreve o processo de implementação de uma estratégia de escalabilidade automática e sob demanda do componente de de visualização do modelo.  

%Estrutura do trabalho
A Seção 2 deste artigo descreve a arquitetura desenvolvida, apresentando os componentes de \textit{software} que a compõe, bem como as motivações para tais escolhas. Posteriormente apresentam-se as ferramentas que darão suporte a esse processo de auto escalabilidade do serviço de visualização e, por fim, a implantação e os resultados da melhoria proposta são apresentados.