
\section{Introdução}
%Contexto geral de transformação
A maturidade alcançada por tecnologias como \textit{big data}, inteligência artificial, intenet das coisas, computação em nuvem, dentre muitas outras, e a redução exponencial dos custo de infra estrutura de TI, vem viabilizando e impulsionando um movimento de transformação digital em diversos ramos e segmentos de negócio.
%TODO: buscar referência, talvez Prof Veras

As empresas, sejam elas públicas ou privadas, de pequeno ou grande porte, em sua maioria já operam com o auxílio de um conjunto de soluções de \textit{sofware} que armazenam os dados de negócio, gerados tanto pela própria companhia quanto por clientes e parceiros. Somam-se a esses dados, as informações publicamente disponíveis na internet e esse conjunto de dados escondem informações valiosíssimas que se bem utilizadas podem significar a sucesso ou o fracasso de um negócio. 

Aliando-se, então, a disponibilidade dos dados,a redução no custo dos insumos de TI e maturidade atingida por novas tecnologias, vem impulsionando as iniciativas de desenvolvimento e implantação de soluções de dados por meio da aplicação de técnicas como \textit{business intelligence}, que consiste na aquisição, tratamento de dados de negócio e armazenamento de dados de negócio, bem como o processo analítico sobre esses dados com o intuito de descoberta de padrões e tendências.
%TODO: buscar referência de definição de BI e possível ajuste 

%Iniciativa da Residência no TRE
Dentro deste contexto, uma das iniciativas do Tribunal Regional Eleitoral do Rio Grande do Norte para fomentar o desenvolvimento de solução de dados foi a implantação de uma turma de residência em tecnologia da informação em parceria com a Universidade Federal do Rio Grande do Norte, onde uma das enfâses a serem trabalhas era exatamente a de \textit{business intelligence}. Nesse formato de residência em TI, alunos de pós graduação são alocados em projetos transformacionais dentro da instituição parceiro para ideação e desenvolvimento de soluções de cunho tecnológico.

%Objetivo do BI
Tratando-se especificamente da vertente de BI da residência do TRE, por não possuir modelo prévio nem ferramentas voltadas para o desenvolvimento de produtos deste tipo, o objetivo principal da iniciativa foi a elaboração e implantação de um modelo de referência para desenvolvimento dos produtos composto por componentes de \textit{software} livre. 

%Arquitetura
O objetivo principal foi alcançado e se concretiza em um conjunto de soluções gratuitas que juntas suportam todo o processo de desenvolvimento de uma aplicação de \textit{business intelligence}. Estas soluções são hospedados por meio de containeres \textit{Docker}.

Com o intuito de garantir mais robustez e escabilidade a esse modelo de referência, este trabalho descreve o processo de implementação de uma estratégia de escalabilidade automática e sob demanda de uma dos componentes desse modelo de referência desenvolvido durante a residência de TI do Tribunal Regional Eleitoral do Rio Grande do Norte.  

%Estrutura do trabalho
A sessão dois deste artigo descreve o modelo de referência desenvolvido, apresentando os componentes de \textit{software} que a compõe, bem como as motivações para suas escolhas. Posteriormente apresentam-se as ferramentas que darão suporte a esse processo de auto escalabilidade do serviço de visualização e, por fim, a implantação e os resultados dessa nova arquitetura são apresentados.