\section{Estudo de Caso}

% Reafirmar o objetivo
O modelo de referência descrito na secão anterior, fruto da iniciativa de BI do Tribunal Regional do Rio Grande do Norta ainda não foi massivamente disponibilizado para os servidores da instituição, até o momento da escrita deste trabalho. Logo, a carga ao qual os serviços que compõem o modelo de referência ainda não é tão elevada, não apresentando muitos momentos de degradação de performance.

No entanto, durante o ano de 2019 ocorreram alguns eventos de divulgação e repasse dos trabalhos desenvolvidos pela residência de TI, onde poderam-se observar momentos de carga não habitual, principalmente no serviço de visualização de dados, o Metabase.

Um destes momentos foi um treinamento de cerca de 60 servidores de diversos Tribunais da Justiça Eleitoral brasileira que vieram a sede do TRE-RN, onde foram ministrados dois dias de conteúdo prático de desenvolvimento de aplicações de dados utilizando-se o modelo de referência. Os efeitos sentidos pelos servidores convidados foram basicamente a indisponibilidade do serviço ou tempo de resposta maior que o habitual.

Infelizmente, as estatíscas de consumo da instância do Metabase que foi utilizada para este treinamento não foram salvas, por se tratar de uma instância criada com o único objetivo de realização do evento. Para que possamos quantificar o impacto do cenário descrito, tentou-se simular a carga a qual o Metabase foi exposto.

A ferramenta utilizada para a simulação foi o \textit{Apache HTTP server benchmarking}, utilizando um \textit{MacBook Pro Mid 2014} equipado com processador Intel Core i5-4278U, 8GB de RAM DDR3 e rodando o sistema macOS Mojave versão 10.14.6. O serviço do Metabase foi hospedado no mesmo modelo descrito na Figura \ref{fig:arq_metabase}, utilizando containeres \textit{Docker}. 