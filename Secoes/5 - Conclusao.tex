\section{Conclusão}

Esse artigo teve como objetivo descrever a análise de ferramentas de gerenciamento de workflow que pudessem ser aplicadas no TRE-RN. Com o aumento da quantidade de processos de ETL, o gerenciamento e manutenção desses processos começa a se tornar cada vez mais complexo utilizando o Cron. Com esse problema em vista, pensamos em utilizar uma ferramenta de gerenciamento de workflow para que pudesse sanar esses problemas através da uma interface mais amigável e de melhor gerenciamento.

Com o uso da ferramenta, foi observamos que o processo de organização desses ETLs se tornou mais fácil, como também o seu gerenciamento. Isso se deu principalmente graças às funcionalidades que o Airflow possui, como a interface gráfica extremamente robusta e interativa.

A partir do que descrevemos nesse artigo, novos caminhos podem ser tomados para melhorar a aplicação, como a definição de padrões na criação dos agendamentos, para que o processo seja mais fácil de dar manutenção. Além disso, fazer uso da arquitetura em multi nós para que o desempenho das jobs sejam cada vez mais eficientes através da distribuição das tarefas através dos workers.