%%%%%%%% ICML 2019 EXAMPLE LATEX SUBMISSION FILE %%%%%%%%%%%%%%%%%

\documentclass{article}

% Recommended, but optional, packages for figures and better typesetting:
\usepackage[brazilian]{babel}
\usepackage[utf8]{inputenc}
\usepackage[T1]{fontenc}

\usepackage{microtype}
\usepackage{graphicx}
\usepackage{subfigure}
\usepackage{booktabs} % for professional tables
\usepackage{import}
\usepackage{array}
\usepackage{pifont}
\newcommand{\cmark}{\ding{51}}
\newcommand{\xmark}{\ding{55}}

% hyperref makes hyperlinks in the resulting PDF.
% If your build breaks (sometimes temporarily if a hyperlink spans a page)
% please comment out the following usepackage line and replace
% \usepackage{icml2019} with \usepackage[nohyperref]{icml2019} above.
\usepackage{hyperref}

% Attempt to make hyperref and algorithmic work together better:
\newcommand{\theHalgorithm}{\arabic{algorithm}}

% Use the following line for the initial blind version submitted for review:
\usepackage[accepted]{icml2019}

% If accepted, instead use the following line for the camera-ready submission:
%\usepackage[accepted]{icml2019}

% The \icmltitle you define below is probably too long as a header.
% Therefore, a short form for the running title is supplied here:
\icmltitlerunning{Gerenciamento de workflow para agendamento de ETL no TRE-RN}

\begin{document}

\twocolumn[
\icmltitle{Análise de ferramentas de gerenciamento de workflow para agendamento de ETL no TRE-RN}

% It is OKAY to include author information, even for blind
% submissions: the style file will automatically remove it for you
% unless you've provided the [accepted] option to the icml2019
% package.

% List of affiliations: The first argument should be a (short)
% identifier you will use later to specify author affiliations
% Academic affiliations should list Department, University, City, Region, Country
% Industry affiliations should list Company, City, Region, Country

% You can specify symbols, otherwise they are numbered in order.
% Ideally, you should not use this facility. Affiliations will be numbered
% in order of appearance and this is the preferred way.
\icmlsetsymbol{equal}{*}

\begin{icmlauthorlist}
\icmlauthor{Thiago de Oliveira}{equal,to}
\end{icmlauthorlist}

\icmlaffiliation{to}{Insituto Metrópole Digital, Universidade Federal do Rio Grande do Norte, Rio Grande do Norte, Brasil}

\icmlcorrespondingauthor{Thiago de Oliveira}{thiago7600@gmail.com}

% You may provide any keywords that you
% find helpful for describing your paper; these are used to populate
% the "keywords" metadata in the PDF but will not be shown in the document
\icmlkeywords{Machine Learning, ICML}

\vskip 0.3in
]

% this must go after the closing bracket ] following \twocolumn[ ...

% This command actually creates the footnote in the first column
% listing the affiliations and the copyright notice.
% The command takes one argument, which is text to display at the start of the footnote.
% The \icmlEqualContribution command is standard text for equal contribution.
% Remove it (just {}) if you do not need this facility.

\printAffiliationsAndNotice{}  % leave blank if no need to mention equal contribution
% \printAffiliationsAndNotice{\icmlEqualContribution} % otherwise use the standard text.

\begin{abstract}
% Esse artigo apresenta uma análise de ferramentas de workflow para o Tribunal Regional Eleitoral(TRE). A partir da observação da dificuldade de se gerenciar workflows no TRE, verificou-se a necessidade do uso de uma ferramenta que pudesse auxiliar esse processo. A partir disso, elencou-se ferramentas que poderiam satisfazer esse tipo de necessidade e selecionada uma ferramenta com base na sua usabilidade. Com esta publicação, espera-se que seja possível gerenciar de forma mais fácil os workflows existentes no TRE.

% - gerenciamento de workflows
% - contexto do TRE-RN
% - proposta
% - resultados

O gerenciamento de \textit{workflows} no Tribunal Regional Eleitoral do Rio Grande do Norte (TRE-RN) apresenta desafios crescentes, dentre os quais a dificuldade de monitoramento, depuração e alteração. Neste trabalho, analisamos ferramentas de gerenciamento de workflow a partir de critérios de adequação ao contexto do TRE-RN. A partir desta análise, avaliamos o uso do Apache Airflow usando diferentes arquiteturas. Os resultados obtidos indicam que uma destas arquiteturas ajuda a mitigar os desafios de gerenciamento de \textit{workflows} no TRE-RN.
\end{abstract}
\import{Secoes/}{01Intro.tex}

\import{Secoes/}{02Contexto.tex}

\import{Secoes/}{03Ferramentas.tex}

\import{Secoes/}{04Solucao.tex}

\import{Secoes/}{5 - Conclusao.tex}


% In the unusual situation where you want a paper to appear in the
% references without citing it in the main text, use \nocite
\nocite{modelagem}
\nocite{metabase}
\nocite{oozieegit}
\nocite{ooziewebsite}
\nocite{oozietuto}
\nocite{auroradocs}
\nocite{auroragit}
\nocite{mesosdoc}
\nocite{mesoswiki}
\nocite{azakbandocs}
\nocite{azkabangit}
\nocite{luigidocs}
\nocite{luigigit}
\nocite{airflowarc}
\nocite{airflowarc2}
\nocite{airflowdocs}
\nocite{airflowgit}

\bibliography{example_paper}
\bibliographystyle{icml2019}


\end{document}


% This document was modified from the file originally made available by
% Pat Langley and Andrea Danyluk for ICML-2K. This version was created
% by Iain Murray in 2018, and modified by Alexandre Bouchard in
% 2019. Previous contributors include Dan Roy, Lise Getoor and Tobias
% Scheffer, which was slightly modified from the 2010 version by
% Thorsten Joachims & Johannes Fuernkranz, slightly modified from the
% 2009 version by Kiri Wagstaff and Sam Roweis's 2008 version, which is
% slightly modified from Prasad Tadepalli's 2007 version which is a
% lightly changed version of the previous year's version by Andrew
% Moore, which was in turn edited from those of Kristian Kersting and
% Codrina Lauth. Alex Smola contributed to the algorithmic style files.
