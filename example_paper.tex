\documentclass{article}

\usepackage[brazilian]{babel}
\usepackage[utf8]{inputenc}
\usepackage[T1]{fontenc}

\usepackage{microtype}
\usepackage{graphicx}
\usepackage{subfigure}
\usepackage{booktabs} % for professional tables
\usepackage{import}
\usepackage{array}
\usepackage{pifont}
\newcommand{\cmark}{\ding{51}}
\newcommand{\xmark}{\ding{55}}

\usepackage{hyperref}

\newcommand{\theHalgorithm}{\arabic{algorithm}}

\usepackage[accepted]{icml2019}

\icmltitlerunning{Implantação de Serviço Auto Escalável para Visualização de Dados Utilizando Containeres}

\begin{document}

\twocolumn[
\icmltitle{Implantação de Serviço Auto Escalável para Visualização de Dados Utilizando Containeres}

\icmlsetsymbol{equal}{*}

\begin{icmlauthorlist}
\icmlauthor{Filipe Viana Monteiro}{equal,to}
\end{icmlauthorlist}

\icmlaffiliation{to}{Insituto Metrópole Digital, Universidade Federal do Rio Grande do Norte, Rio Grande do Norte, Brasil}

\icmlcorrespondingauthor{Filipe Viana Monteiro}{filipevianam@gmail.com}

% You may provide any keywords that you
% find helpful for describing your paper; these are used to populate
% the "keywords" metadata in the PDF but will not be shown in the document
\icmlkeywords{Auto Scaling, Docker, Kubernetes, Arquitetura de Software}

\vskip 0.3in
]
\printAffiliationsAndNotice{}  % leave blank if no need to mention equal contribution

\begin{abstract}
A competitividade do mercado e a busca por operações mais lucrativas são uns dos motivos que vêm obrigando as empresas a realizar decisões de negócio mais assertivas e em espaços de tempo cada vez menores e é nesse contexto que a utilização de dados vem se mostrando de grande valia, impulsionando a adesão de soluções de \textit{business intelligence}. Essas soluções viabilizam a tomada de decisões embasadas em dados gerados pela empresa, seus clientes e até mesmo entidades externas ao negócio. Com o intuito de se adequar a essa realidade, uma das ações realizadas pelo Tribunal Regional Eleitoral do Rio Grande do Norte, foi a implantação de um residência de TI com ênfase em BI, em parceria com a Universidade Federal do Rio Grande do Norte. Um dos produtos dessa residência foi o desenvolvimento de uma arquitetura de \textit{software} capaz de implementar soluções de BI, abrangendo as etapas de transformação, armazenamento e visualização de dados. Esse trabalho tem por objetivo propor um aperfeiçoamento nessa arquitetura para garantir maior escalabilidade e disponibilidade do serviço responsável pela camada de visualização, através da implantação de um \textit{cluster} auto escalável de containers, onde será hospedado o serviço em questão.
\end{abstract}

\import{Secoes/}{01Intro.tex}

\import{Secoes/}{02ModelRef.tex}

%\import{Secoes/}{03Ferramentas.tex}

%\import{Secoes/}{04Solucao.tex}

%\import{Secoes/}{5 - Conclusao.tex}


% In the unusual situation where you want a paper to appear in the
% references without citing it in the main text, use \nocite

%\nocite{modelagem}
%\nocite{metabase}


%\bibliography{example_paper}
\bibliographystyle{icml2019}


\end{document}
